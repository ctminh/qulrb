%!TEX root = main.tex

\section{Related Work} \label{sec:relatedwork}

% Topic and problem overview:
%   + load balancing definition, static vs dynamic
%   + when it happens, and how can we balance it again

Load rebalancing problem is formally defined by \cite{aggarwal2003lrb} in the context of job processing in multiprocessor systems. We have a given sequence of jobs that are assigned to $m$ processors. However, in most of real world scenarios with a dynamic measure, the given assignment of jobs might not be optimal and balanced during execution. Load rebalancing is motivated by relocating a set of jobs from proper processors to rebalance the overall performance as well as decrease the makespan. \\

In the common sense, if load values can be known in advance, the problem is originally considered as an optimization problem, i.e., number partitioning, knapsack problem. However, when and where jobs are relocated defines the problem constraints in a specific context. For examples, traditional number partitioning, knapsacks comes with a set of numbers or jobs with weights, then the requirement is a proper assignment of numbers or jobs to processors before execution. Some of the realistic use cases in HPC are Adaptive Mesh Refinement and Smoothed Particle Hydrodynamics. Given a mesh, computation units reflects to the cells of a deterministic mesh. Each cell contains a collection of computation workload, therefore, partitioning the cells with estimated load values indicates number partitioning problem. Rathore et al. propose a new approach using quantum annealing to solve the problem targeting load balancing for HPC \cite{rathore2024lbhpc}.