%%
%% This is file `sample-sigconf-authordraft.tex',
%% generated with the docstrip utility.
%%
%% The original source files were:
%%
%% samples.dtx  (with options: `all,proceedings,bibtex,authordraft')
%% 
%% IMPORTANT NOTICE:
%% 
%% For the copyright see the source file.
%% 
%% Any modified versions of this file must be renamed
%% with new filenames distinct from sample-sigconf-authordraft.tex.
%% 
%% For distribution of the original source see the terms
%% for copying and modification in the file samples.dtx.
%% 
%% This generated file may be distributed as long as the
%% original source files, as listed above, are part of the
%% same distribution. (The sources need not necessarily be
%% in the same archive or directory.)
%%
%%
%% Commands for TeXCount
%TC:macro \cite [option:text,text]
%TC:macro \citep [option:text,text]
%TC:macro \citet [option:text,text]
%TC:envir table 0 1
%TC:envir table* 0 1
%TC:envir tabular [ignore] word
%TC:envir displaymath 0 word
%TC:envir math 0 word
%TC:envir comment 0 0
%%
%%
%% The first command in your LaTeX source must be the \documentclass
%% command.
%%
%% For submission and review of your manuscript please change the
%% command to \documentclass[manuscript, screen, review]{acmart}.
%%
%% When submitting camera ready or to TAPS, please change the command
%% to \documentclass[sigconf]{acmart} or whichever template is required
%% for your publication.
%%
%%
\documentclass[sigconf,authordraft]{acmart}

%%
%% \BibTeX command to typeset BibTeX logo in the docs
\AtBeginDocument{%
  \providecommand\BibTeX{{%
    Bib\TeX}}}

%% Rights management information.  This information is sent to you
%% when you complete the rights form.  These commands have SAMPLE
%% values in them; it is your responsibility as an author to replace
%% the commands and values with those provided to you when you
%% complete the rights form.
\setcopyright{acmlicensed}
\copyrightyear{2024}
\acmYear{2024}
\acmDOI{XXXXXXX.XXXXXXX}

%% These commands are for a PROCEEDINGS abstract or paper.
\acmConference[Conference acronym 'XX]{Make sure to enter the correct
  conference title from your rights confirmation emai}{June 25--30,
  2024}{Woodstock, NY}
%%
%%  Uncomment \acmBooktitle if the title of the proceedings is different
%%  from ``Proceedings of ...''!
%%
%%\acmBooktitle{Woodstock '18: ACM Symposium on Neural Gaze Detection,
%%  June 03--05, 2018, Woodstock, NY}
\acmISBN{978-1-4503-XXXX-X/18/06}


%%
%% Submission ID.
%% Use this when submitting an article to a sponsored event. You'll
%% receive a unique submission ID from the organizers
%% of the event, and this ID should be used as the parameter to this command.
%%\acmSubmissionID{123-A56-BU3}

%%
%% For managing citations, it is recommended to use bibliography
%% files in BibTeX format.
%%
%% You can then either use BibTeX with the ACM-Reference-Format style,
%% or BibLaTeX with the acmnumeric or acmauthoryear sytles, that include
%% support for advanced citation of software artefact from the
%% biblatex-software package, also separately available on CTAN.
%%
%% Look at the sample-*-biblatex.tex files for templates showcasing
%% the biblatex styles.
%%

%%
%% The majority of ACM publications use numbered citations and
%% references.  The command \citestyle{authoryear} switches to the
%% "author year" style.
%%
%% If you are preparing content for an event
%% sponsored by ACM SIGGRAPH, you must use the "author year" style of
%% citations and references.
%% Uncommenting
%% the next command will enable that style.
%%\citestyle{acmauthoryear}


%%
%% end of the preamble, start of the body of the document source.
\begin{document}

%%
%% The "title" command has an optional parameter,
%% allowing the author to define a "short title" to be used in page headers.
\title{Hybrid Classical-Quantum Load Rebalancing}

%%
%% The "author" command and its associated commands are used to define
%% the authors and their affiliations.
%% Of note is the shared affiliation of the first two authors, and the
%% "authornote" and "authornotemark" commands
%% used to denote shared contribution to the research.
\author{RX1 RY1 RZ1}
\authornote{Both authors contributed equally to this research.}
\email{rxryrz1@mail.com}
\orcid{1234-5678-91011}
\author{RX2 RY2 RZ2}
\authornotemark[1]
\email{rxryrz2@mail.com}
\affiliation{%
  \institution{Institute for Clarity in Documentation}
  \city{Krakow}
  \country{Poland}
}

\author{RX3 RY3 RZ3}
\affiliation{%
  \institution{Institute for Clarity in Documentation}
  \city{Munich}
  \country{Germany}}
\email{rxryrz3@mail.com}

\author{RX4 RY4 RZ4}
\affiliation{%
  \institution{Institute for Clarity in Documentation}
  \city{Munich}
  \country{Germany}
}
\email{rxryrz4@mail.com}


%%
%% By default, the full list of authors will be used in the page
%% headers. Often, this list is too long, and will overlap
%% other information printed in the page headers. This command allows
%% the author to define a more concise list
%% of authors' names for this purpose.
\renewcommand{\shortauthors}{RX RY RZ et al.}

%%
%% The abstract is a short summary of the work to be presented in the
%% article.
\begin{abstract}
  A clear and well-documented \LaTeX\ document is presented as an
  article formatted for publication by ACM in a conference proceedings
  or journal publication. Based on the ``acmart'' document class, this
  article presents and explains many of the common variations, as well
  as many of the formatting elements an author may use in the
  preparation of the documentation of their work.
\end{abstract}

%%
%% The code below is generated by the tool at http://dl.acm.org/ccs.cfm.
%% Please copy and paste the code instead of the example below.
%%
\begin{CCSXML}
<ccs2012>
 <concept>
  <concept_id>00000000.0000000.0000000</concept_id>
  <concept_desc>Do Not Use This Code, Generate the Correct Terms for Your Paper</concept_desc>
  <concept_significance>500</concept_significance>
 </concept>
 <concept>
  <concept_id>00000000.00000000.00000000</concept_id>
  <concept_desc>Do Not Use This Code, Generate the Correct Terms for Your Paper</concept_desc>
  <concept_significance>300</concept_significance>
 </concept>
 <concept>
  <concept_id>00000000.00000000.00000000</concept_id>
  <concept_desc>Do Not Use This Code, Generate the Correct Terms for Your Paper</concept_desc>
  <concept_significance>100</concept_significance>
 </concept>
 <concept>
  <concept_id>00000000.00000000.00000000</concept_id>
  <concept_desc>Do Not Use This Code, Generate the Correct Terms for Your Paper</concept_desc>
  <concept_significance>100</concept_significance>
 </concept>
</ccs2012>
\end{CCSXML}

\ccsdesc[500]{Do Not Use This Code~Generate the Correct Terms for Your Paper}
\ccsdesc[300]{Do Not Use This Code~Generate the Correct Terms for Your Paper}
\ccsdesc{Do Not Use This Code~Generate the Correct Terms for Your Paper}
\ccsdesc[100]{Do Not Use This Code~Generate the Correct Terms for Your Paper}

%%
%% Keywords. The author(s) should pick words that accurately describe
%% the work being presented. Separate the keywords with commas.
\keywords{Do, Not, Us, This, Code, Put, the, Correct, Terms, for,
  Your, Paper}

\received{20 February 2007}
\received[revised]{12 March 2009}
\received[accepted]{5 June 2009}

%%
%% This command processes the author and affiliation and title
%% information and builds the first part of the formatted document.
\maketitle

% ------------------------------------------------------------------
% 1. Introduction
% ------------------------------------------------------------------
%!TEX root = main.tex

\section{Introduction} \label{sec:intro}

Task-based or task-graph programming models have been used since the early computers. However, the idea was applied for the whole jobs running in a system rather than decomposing into small tasks of a single application. Recently, compared to shared-memory parallel programming and message-passing interface, they are sometimes tedious and difficult in dealing with concurrency issues. While task-based parallel programming ideas are established on a higher abstraction, the runtimes combine multiprocessing and multithreading. This means in the scope of node-level we deploy multithreading for parallelism (shared-memory). Over nodes, multiprocessing is used to communicate in distributed memory. Furthermore, this programming model is towards abstracting parallel application into tasks, while parallelism does not need much intervention by developers. We need to determine what is a task, then the task-based library would control the rest involved between scheduling and parallelizing of tasks.


% ------------------------------------------------------------------
% 2. Related work
% ------------------------------------------------------------------
%!TEX root = main.tex

\section{Related Work} \label{sec:relatedwork}

% Topic and problem overview:
%   + load balancing definition, static vs dynamic
%   + when it happens, and how can we balance it again

Load rebalancing problem is formally defined by \cite{aggarwal2003lrb} in the context of job processing in multiprocessor systems. We have a given sequence of jobs that are assigned to $m$ processors. However, in most of real world scenarios with a dynamic measure, the given assignment of jobs might not be optimal and balanced during execution. Load rebalancing is motivated by relocating a set of jobs from proper processors to rebalance the overall performance as well as decrease the makespan. \\

In the common sense, if load values can be known in advance, the problem is originally considered as an optimization problem, i.e., number partitioning, knapsack problem. However, when and where jobs are relocated defines the problem constraints in a specific context. For examples, traditional number partitioning, knapsacks comes with a set of numbers or jobs with weights, then the requirement is a proper assignment of numbers or jobs to processors before execution. Some of the realistic use cases in HPC are Adaptive Mesh Refinement and Smoothed Particle Hydrodynamics. Given a mesh, computation units reflects to the cells of a deterministic mesh. Each cell contains a collection of computation workload, therefore, partitioning the cells with estimated load values indicates number partitioning problem. Rathore et al. propose a new approach using quantum annealing to solve the problem targeting load balancing for HPC \cite{rathore2024lbhpc}.

% ------------------------------------------------------------------
% 3. Problem statement
% ------------------------------------------------------------------
%!TEX root = main.tex

\section{Problem Statement} \label{sec:probstate}
% Topic and problem overview:
%   + load balancing definition, static vs dynamic
%   + when it happens, and how can we balance it again

The ``\verb|acmart|'' document class includes the ``\verb|booktabs|''
package --- \url{https://ctan.org/pkg/booktabs} --- for preparing
high-quality tables.

Table captions are placed {\itshape above} the table.

Because tables cannot be split across pages, the best placement for
them is typically the top of the page nearest their initial cite.  To
ensure this proper ``floating'' placement of tables, use the
environment \textbf{table} to enclose the table's contents and the
table caption.  The contents of the table itself must go in the
\textbf{tabular} environment, to be aligned properly in rows and
columns, with the desired horizontal and vertical rules.  Again,
detailed instructions on \textbf{tabular} material are found in the
\textit{\LaTeX\ User's Guide}.

Immediately following this sentence is the point at which
Table~\ref{tab:freq} is included in the input file; compare the
placement of the table here with the table in the printed output of
this document.

\begin{table}
  \caption{Frequency of Special Characters}
  \label{tab:freq}
  \begin{tabular}{ccl}
    \toprule
    Non-English or Math&Frequency&Comments\\
    \midrule
    \O & 1 in 1,000& For Swedish names\\
    $\pi$ & 1 in 5& Common in math\\
    \$ & 4 in 5 & Used in business\\
    $\Psi^2_1$ & 1 in 40,000& Unexplained usage\\
  \bottomrule
\end{tabular}
\end{table}

% ------------------------------------------------------------------
% 4. QUBO formulation
% ------------------------------------------------------------------
%!TEX root = main.tex

\section{QUBO Forumulation} \label{sec:qubo}
% Topic and problem overview:
%   + load balancing definition, static vs dynamic
%   + when it happens, and how can we balance it again

Your work should use standard \LaTeX\ sectioning commands:
\verb|section|, \verb|subsection|, \verb|subsubsection|, and
\verb|paragraph|. They should be numbered; do not remove the numbering
from the commands.

Simulating a sectioning command by setting the first word or words of
a paragraph in boldface or italicized text is {\bfseries not allowed.}

% ------------------------------------------------------------------
% 5. Experiments
% ------------------------------------------------------------------
%!TEX root = main.tex

\section{Experiments} \label{sec:experiments}

The IEEEtran class file is used to format your paper and style the text. All margins, column widths, line spaces, and text fonts are prescribed; please do not  alter them. You may note peculiarities. For example, the head margin
measures proportionately more than is customary. This measurement 
and others are deliberate, using specifications that anticipate your paper 
as one part of the entire proceedings, and not as an independent document. 
Please do not revise any of the current designations.


% ------------------------------------------------------------------
% 6. Discussion
% ------------------------------------------------------------------
%!TEX root = main.tex

\section{Discussion} \label{sec:discussion}
% Topic and problem overview:
%   + load balancing definition, static vs dynamic
%   + when it happens, and how can we balance it again

Each author must be defined separately for accurate metadata
identification.  As an exception, multiple authors may share one
affiliation. Authors' names should not be abbreviated; use full first
names wherever possible. Include authors' e-mail addresses whenever
possible.

Grouping authors' names or e-mail addresses, or providing an ``e-mail
alias,'' as shown below, is not acceptable:
\begin{verbatim}
  \author{Brooke Aster, David Mehldau}
  \email{dave,judy,steve@university.edu}
  \email{firstname.lastname@phillips.org}
\end{verbatim}

The \verb|authornote| and \verb|authornotemark| commands allow a note
to apply to multiple authors --- for example, if the first two authors
of an article contributed equally to the work.

If your author list is lengthy, you must define a shortened version of
the list of authors to be used in the page headers, to prevent
overlapping text. The following command should be placed just after
the last \verb|\author{}| definition:
\begin{verbatim}
  \renewcommand{\shortauthors}{McCartney, et al.}
\end{verbatim}
Omitting this command will force the use of a concatenated list of all
of the authors' names, which may result in overlapping text in the
page headers.

The article template's documentation, available at
\url{https://www.acm.org/publications/proceedings-template}, has a
complete explanation of these commands and tips for their effective
use.

Note that authors' addresses are mandatory for journal articles.

% ------------------------------------------------------------------
% 7. Conclusion
% ------------------------------------------------------------------
%!TEX root = main.tex

\section{Quantum Transformation for Load Rebalancing Problem} \label{sec:lrb_quantum_trans}

The IEEEtran class file is used to format your paper and style the text. All margins, column widths, line spaces, and text fonts are prescribed; please do not  alter them. You may note peculiarities. For example, the head margin
measures proportionately more than is customary. This measurement 
and others are deliberate, using specifications that anticipate your paper 
as one part of the entire proceedings, and not as an independent document. 
Please do not revise any of the current designations.



\section{Acknowledgments}

Identification of funding sources and other support, and thanks to
individuals and groups that assisted in the research and the
preparation of the work should be included in an acknowledgment
section, which is placed just before the reference section in your
document.


%%
%% The next two lines define the bibliography style to be used, and
%% the bibliography file.
\bibliographystyle{ACM-Reference-Format}
\bibliography{references}


\end{document}
\endinput
%%
%% End of file `sample-sigconf-authordraft.tex'.
